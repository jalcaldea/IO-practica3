\section{ANEXO}
Uno de los requisitos especificados para la instalación del nuevo \emph{``Call Center''} es que los servidores tienen que tener una tasa de utilización inferior al 85\%. Para que esta condición se cumpla, ha de cumplirse que:

\begin{equation}
p \leq 0,85 \text{ \texttt{siendo:} }
p = \frac{\lambda_{n}}{m\mu_{n}} \text{ \texttt{y} } \mu_{n}= \frac{1}{\overline{X}_{n}}
\end{equation}

Para $m= numero\ de\ operadores$, $\mu_{n} = tasa\ de\ servicio\ n$, $\overline{X}_{n} = tiempo\ medio\ del\ servicio\ n$ y $\lambda_{n} = tasa\ de\ llegada\ del\ servicio\ n$.\\
Por lo que para calcular el número de operadores para cada servicio, basta con:

\begin{equation}
p = \frac{\lambda_{n}}{m\mu_{n}} \rightarrow p = \frac{\lambda_{n}\overline{X}_{n}}{m} \rightarrow m = \frac{\lambda_{n}\overline{X}_{n}}{p}
\end{equation}

\subsection{Cálculo de las ecuaciones de tráfico}
Para poder poner en práctica las formula anterior necesitamos saber las tasas de llegada de cada servicio. Para los tres primeros estos datos se corresponden con $\lambda_{1}$, $\lambda_{2}$ y $\lambda_{3}$, respectivamente.\\

Para los demás es necesario saber qué porcentaje de llegadas procede da cada lugar, quedándonos lo siguiente:

\begin{multline}\\
  I_{1} = \lambda_{1} \rightarrow I_{1} = 15 \\
  I_{2} = \lambda_{2} \rightarrow I_{2} = 20 \\
  I_{3} = \lambda_{3} \rightarrow I_{3} = 3 \\
  I_{4} = 0.8\cdot (I_{1}+I_{2}+I_{3}) + 0.1I_{6} \rightarrow I_{4} = 30.476 \\
  I_{5} = 0.1\cdot (I_{1}+I_{2}+I_{3})  \rightarrow I_{5} = 3.8 \\
  I_{6} = 0.02\cdot (I_{1}+I_{2}+I_{3})  \rightarrow I_{6} = 0.76 \\
  I_{7} = 0.08\cdot (I_{1}+I_{2}+I_{3})  \rightarrow I_{7} = 3.192 \\
  \end{multline}

Una vez con estos datos podemos calcular el número de operadores.

\subsection{Cálculo del número de operadores}
Eso significa que han de cumplirse estos valores:

\begin{multline}\\
  m_{1} = \frac{\lambda_{1}\overline{X}_{1}}{0.85} \rightarrow m_{1} = \frac{15\cdot 0.25}{0.85} \rightarrow m_{1} = 4.41 \\
  m_{2} = \frac{\lambda_{2}\overline{X}_{2}}{0.85} \rightarrow m_{2} = \frac{20\cdot 0.16}{0.85} \rightarrow m_{2} = 3.92\\
  m_{3} = \frac{\lambda_{3}\overline{X}_{3}}{0.85} \rightarrow m_{3} = \frac{3\cdot 1}{0.85} \rightarrow m_{3} = 3.52\\
  m_{4} = \frac{\lambda_{4}\overline{X}_{4}}{0.85} \rightarrow m_{4} = \frac{30.476\cdot 1}{0.85} \rightarrow m_{4} = 35.85\\
  m_{5} = \frac{\lambda_{5}\overline{X}_{5}}{0.85} \rightarrow m_{5} = \frac{3.8\cdot 3}{0.85} \rightarrow m_{5} = 13.41\\
  m_{6} = \frac{\lambda_{6}\overline{X}_{6}}{0.85} \rightarrow m_{6} = \frac{0.76\cdot 4}{0.85} \rightarrow m_{6} = 3.57\\
  m_{7} = \frac{\lambda_{7}\overline{X}_{7}}{0.85} \rightarrow m_{7} = \frac{3.192\cdot 5}{0.85} \rightarrow m_{7} = 18.77\\
\end{multline}

Como no se tratan de valores enteros y éstos deben de ser los valores mínimos, el número mínimo de operadores ha de ser:

\begin{table}[H]
  \begin{center}
  \begin{tabular}{|c|c|}
    \hline
    \textbf{Servicio}       & \textbf{Operadores} \\ \hline
    Llamadas por Teléfono   & 5                   \\ \hline
    Peticiones por Internet & 4                   \\ \hline
    Peticiones por FAX      & 4                   \\ \hline
    Consulta Facturas      & 36                   \\ \hline
    Nuevos Clientes      & 14                   \\ \hline
    Reclamaciones      & 4                   \\ \hline
    Servicio Técnico      & 19                   \\ \hline
  \end{tabular}
\end{center}
  \caption{Numero de operadores mínimos para lograr una tasa de utilización inferior al 85\%}
\end{table}

\subsection{Cálculo de la tasa de uso de cada servidor}
Como ya se ha indicado anteriormente, la tasa de uso de cada servidor la obtenemos mediante la fórmula:
\begin{equation}
p = \frac{\lambda_{n}}{m\mu_{n}}
\end{equation}
Por tanto, insertamos el número de operarios necesarios calculados y obtenemos la tasa de uso de cada servidor:
\begin{multline}\\
p_{1} = \frac{\lambda_{1}}{m_{1}\mu_{1}} = \frac{15}{5\times 4} = 0,75 \\
p_{2} = \frac{\lambda_{2}}{m_{2}\mu_{2}} = \frac{20}{4\times 6} = 0,8\\
p_{3} = \frac{\lambda_{3}}{m_{3}\mu_{3}} = \frac{3}{4\times 1} = 0,75\\
p_{4} = \frac{\lambda_{4}}{m_{4}\mu_{4}} = \frac{30,476}{36\times 1} = 0,8465\\
p_{5} = \frac{\lambda_{5}}{m_{5}\mu_{5}} = \frac{3,8}{14\times 0,33} = 0,8225\\
p_{6} = \frac{\lambda_{6}}{m_{6}\mu_{6}} = \frac{0,76}{4\times 0,25} = 0,75\\
p_{7} = \frac{\lambda_{7}}{m_{7}\mu_{7}} = \frac{3,192}{19\times 0,2} = 0,84\\
\end{multline}
Los resultados los agrupamos en la siguiente tabla:
\begin{table}[H]
  \begin{center}
  \begin{tabular}{|c|c|}
    \hline
    \textbf{Servicio}       & \textbf{Tasa de uso} \\ \hline
    Llamadas por Teléfono   & 75\%                   \\ \hline
    Peticiones por Internet & 80\%                  \\ \hline
    Peticiones por FAX      & 75\%                   \\ \hline
    Consulta Facturas      & 84,65\%                   \\ \hline
    Nuevos Clientes      & 82,25\%                   \\ \hline
    Reclamaciones      & 75\%                   \\ \hline
    Servicio Técnico      & 84\%                  \\ \hline
  \end{tabular}
\end{center}
  \caption{Tasa de uso de cada servidor}
  \end{table}
\subsection{Cálculo del tiempo medio de respuesta en cada servidor}

El tiempo medio de respuesta se calcula utilizando la siguiente fórmula:

\begin{equation}
\overline{R}_{i} = \frac{1}{m_{i}\mu_{i}-I_{i}}
\end{equation}

Siendo $i$ el servidor.\\

Sustituyendo en la formula obtenemos los siguientes resultados:

\begin{multline}\\
\overline{R}_{1} = \frac{1}{m_{1}\mu_{1}-I_{1}} = \frac{1}{5\times 4 - 15} = 0,2\\
\overline{R}_{2} = \frac{1}{m_{2}\mu_{2}-I_{2}} = \frac{1}{4\times 6 - 20} = 0,1\\
\overline{R}_{3} = \frac{1}{m_{3}\mu_{3}-I_{3}} = \frac{1}{4\times 1 - 3} = 1\\
\overline{R}_{4} = \frac{1}{m_{4}\mu_{4}-I_{4}} = \frac{1}{36\times 1 - 30,48} = 0,18\\
\overline{R}_{5} = \frac{1}{m_{5}\mu_{5}-I_{5}} = \frac{1}{14\times 0,33 - 3,8} = 0,82\\
\overline{R}_{6} = \frac{1}{m_{6}\mu_{6}-I_{6}} = \frac{1}{4\times 0,25 - 0,76} = 4,17\\
\overline{R}_{7} = \frac{1}{m_{7}\mu_{7}-I_{7}} = \frac{1}{18\times 0,2 - 3,04} = 1,79\\
\end{multline}
Los resultados los agrupamos en la siguiente tabla:

\begin{table}[H]
  \begin{center}
  \begin{tabular}{|c|c|}
    \hline
    \textbf{Servicio}       & \textbf{Tiempo medio de respuesta en cada servidor(m)} \\ \hline
    Llamadas por Teléfono   & 0,2                   \\ \hline
    Peticiones por Internet & 0,1                  \\ \hline
    Peticiones por FAX      & 1                   \\ \hline
    Consulta Facturas      & 0,18                   \\ \hline
    Nuevos Clientes      & 0,82                   \\ \hline
    Reclamaciones      & 4,17                   \\ \hline
    Servicio Técnico      & 1,79                   \\ \hline
  \end{tabular}
\end{center}
  \caption{Tiempo medio de respuesta en cada servidor}
  \end{table}
  
\subsection{Cálculo del tiempo medio de respuesta para cada tipo de trabajo}
El tiempo medio de respuesta para cada tipo de trabajo se calcula multiplicando el tiempo medio de respuesta de cada servidor por la probabilidad de que la solicitud pase por dicho servidor.\\

Es decir:
\begin{multline}\\
\overline{R}' = \overline{R}_{1} + 0,8\overline{R}_{4} + 0,1\overline{R}_{5} + 0,02\overline{R}_{6} + 0,08\overline{R}_{7} + 0,02\times 0,1\overline{R}_{4} + 0,02\times 0,2\overline{R}_{7} = 0,66\\
\overline{R}'' = \overline{R}_{2} + 0,8\overline{R}_{4} + 0,1\overline{R}_{5} + 0,02\overline{R}_{6} + 0,08\overline{R}_{7} + 0,02\times 0,1\overline{R}_{4} + 0,02\times 0,2\overline{R}_{7} = 0,56\\
\overline{R}''' = \overline{R}_{3} + 0,8\overline{R}_{4} + 0,1\overline{R}_{5} + 0,02\overline{R}_{6} + 0,08\overline{R}_{7} + 0,02\times 0,1\overline{R}_{4} + 0,02\times 0,2\overline{R}_{7} = 1,46\\
\end{multline}

\subsection{Cálculo del tiempo medio de respuesta total}
El tiempo medio de respuesta total se calcula mediante la siguiente fórmula:

\begin{equation}
\overline{R}_{T} = \frac{\lambda_{1}}{\lambda_{1}+\lambda_{2}+\lambda_{3}}\times \overline{R}' + \frac{\lambda_{2}}{\lambda_{1}+\lambda_{2}+\lambda_{3}}\times \overline{R}'' + \frac{\lambda_{3}}{\lambda_{1}+\lambda_{2}+\lambda_{3}}\times \overline{R}'''
\end{equation}

Sustituyendo obtenemos el siguiente resultado:

\begin{equation}
\overline{R}_{T} = \frac{15}{15+20+3}\times 0,66 + \frac{20}{15+20+3}\times 0,56 + \frac{3}{15+20+3}\times 1,46 = 0,67
\end{equation}