\section{\textbf{\underline{Conclusiones}}}
Tras la realización de este estudio podemos concluir lo siguiente:
\begin{itemize}
\item Si se dispone de los operadores mínimos indicados anteriormente, el sistema es estable y la saturación de cada estación multiservicio no sobrepasa el 85\%.
\item Los tres primeros servidores (teleoperadoras, programas automáticos y servicios de fax) van a tener una tasa de uso siempre menor al 85\% y no se formarán colas, ya que despachan trabajos a una velocidad mayor de los que los reciben.
\item Sin embargo en los cuatro servidores restantes, al depender del azar, sí se pueden producir colas. Especialmente delicado es el caso del multiservicio dedicado a las reclamaciones: aunque es el servidor con menos tráfico (1,4 \% del total) y trabajar con suficiente margen (75\%), posee un tiempo en cola muy elevado (67,93 segundos) y una desviación estándar alta.
\end{itemize}


