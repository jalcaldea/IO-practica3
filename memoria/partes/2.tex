\section{Estudio del modelo}
Como primer paso, se ha calculado el número mínimo de operarios necesarios en cada servidor para que se garantice un nivel máximo de saturación del 85 \%. Los resultados se adjuntan en la tabla siguiente. Para más información sobre el proceso de obtención de estos datos puede consultarse el \textit{Anexo}.
\begin{table}[H]
  \begin{center}
  \begin{tabular}{|c|c|}
    \hline
    \textbf{Servicio}       & \textbf{Operadores} \\ \hline
    Llamadas por Teléfono   & 5                   \\ \hline
    Peticiones por Internet & 4                   \\ \hline
    Peticiones por FAX      & 4                   \\ \hline
    Consulta Facturas      & 36                   \\ \hline
    Nuevos Clientes      & 14                   \\ \hline
    Reclamaciones      & 4                   \\ \hline
    Servicio Técnico      & 19                   \\ \hline
  \end{tabular}
\end{center}
  \caption{Numero de operadores mínimos para lograr una tasa de utilización inferior al 85\%}
\end{table}
A partir de estos resultados podemos proceder a un estudio más pormenorizado del sistema. Para ello nos serviremos de los siguientes indicadores: tasa de uso de cada servidor, tiempo medio de espera en cada servidor y tiempo medio de respuesta por cada tipo de trabajo y general.
\subsection{Tasa de uso de cada servidor}
La tasa de uso de cada servidor se indica en la siguiente tabla. Para más detalles sobre la obtención de estos resultados puede consultarse el \textit{Anexo}.\\
\begin{table}[H]
  \begin{center}
  \begin{tabular}{|c|c|}
    \hline
    \textbf{Servicio}       & \textbf{Tasa de uso} \\ \hline
    Llamadas por Teléfono   & 75\%                   \\ \hline
    Peticiones por Internet & 80\%                  \\ \hline
    Peticiones por FAX      & 75\%                   \\ \hline
    Consulta Facturas      & 84,65\%                   \\ \hline
    Nuevos Clientes      & 82,25\%                   \\ \hline
    Reclamaciones      & 75\%                   \\ \hline
    Servicio Técnico      & 84\%                  \\ \hline
  \end{tabular}
\end{center}
  \caption{Tasa de uso de cada servidor}
  \end{table}
\subsection{Tiempo medio de espera en cada servidor}
El Tiempo medio de espera en cada servidor se indica en la siguiente tabla. Para más detalles sobre la obtención de estos resultados puede consultarse el \textit{Anexo}.\\
CARLOS DEL FUTURO METE ESTOS DATOS
\begin{table}[H]
  \begin{center}
  \begin{tabular}{|c|c|}
    \hline
    \textbf{Servicio}       & \textbf{Tiempo medio de espera en cada servidor(m)} \\ \hline
    Llamadas por Teléfono   & 9999                   \\ \hline
    Peticiones por Internet & 9999                  \\ \hline
    Peticiones por FAX      & 9999                   \\ \hline
    Consulta Facturas      & 9999                   \\ \hline
    Nuevos Clientes      & 9999                   \\ \hline
    Reclamaciones      & 9999                   \\ \hline
    Servicio Técnico      & 9999                   \\ \hline
  \end{tabular}
\end{center}
  \caption{Tiempo medio de espera en cada servidor}
  \end{table}
\subsection{Tiempo medio de respuesta por cada tipo de trabajo y general}