\section{Introducción}

En este documento se redacta el informe correspondiente al estudio y simulación de un \emph{``CallCenter''}, mendiante el cúal una empresa de telecominicaciones desea implantar un sistema de atención al cliente.\\

En primer lugar vamos a hacer un análisis de dicho sistema.

\subsection{Anális del escenario}
Éste sistema, según nos informa la compañía, puede recibir información a través de tres puntos y que cada uno de éstos es procesado de manera independiente:

\begin{itemize}
  \item \textbf{Llamadas por teléfono:} Se procesan en el \emph{``Servicio de tele-operadores''}. Tienen una media de llegadas de 15 llamadas por minuto ($\lambda_{1}$) y cada petición tarda una media de 15 segundos ($\mu_{1}$).
  \item \textbf{Peticiones vía Internet:} Se procesan mediante \emph{``Programas automáticos''}. Tienen una media de llegadas de 20 peticiones por minuto ($\lambda_{2}$) y cada petición tarda una media de 10 segundos ($\mu_{2}$).
  \item \textbf{Peticiones vía FAX:} Son procesadas por \emph{``Operadores de fax''}. Tienen una media de llegadas de 3 peticiones por minuto ($\lambda_{3}$) y cada petición tarda una media de 60 segundos ($\mu_{3}$).
\end{itemize}

Se explicita que las peticiones serán procesadas en orden de llegada y que esperaran en caso de que el servidor esté ocupado, por lo que suponemos que cada servidor dispone de una cola infinita para recibir peticiones.\\

Una vez que son procesados en los servidores mentados anteriormente, las peticiones se trasladas al servicio que corresponda, siguiendo una proporción similar, éstos son los servicios:

\begin{itemize}
  \item \textbf{Consultas sobre facturas:} Cada petición tiene un tiempo medio de servicio de 1 minuto ($\overline{X}_{4}$). Llegan hasta aquí el 80\% de las peticiones de cada servidor de llegadas más un 10\% de \emph{``reclamaciones de clientes''}.
  \item \textbf{Solicitudes de nuevos clientes:} Con un tiempo medio de servicio de 3 minutos ($\overline{X}_{5}$). Suponen el 10\% de las peticiones de cada servidor de llegadas.
  \item \textbf{Reclamaciones de clientes:} Se tardan 4 minutos de media en atender la petición ($\overline{X}_{6}$). Suponen el 2\% de las peticiones de cada servidor de llegadas.
  \item \textbf{Servicio tecnico :} Cada petición tiene un tiempo medio de servicio de 5 minutos ($\overline{X}_{7}$). Llegan hasta aquí el 8\% de las peticiones de cada servidor de llegadas más un 20\% de \emph{``reclamaciones de clientes''}.
\end{itemize}

\subsection{Representación visual del sistema}
Para poder entender mejor el sistema, con los datos proporcionados anteriormente, se ha esbozado el siguiente esquema del sistema utilizando un diagrama TQM (\emph{``Total Quality Management''}):

\tikzstyle{format} = [draw, thin, fill=blue!20]
\tikzstyle{medium} = [ellipse, draw, thin, fill=green!20, minimum height=2.5em]

\begin{figure}[H]
  \center{
  \begin{tikzpicture}[node distance=1.5cm, auto,>=latex', thick]

    \path[->] node (a1) {};
    \path[->] node[format, right=0cm and 3cm of a1] (s1) {Teléfono}
    (a1) edge node {$\lambda_{1}$} (s1);

    \path[->] node[below of=a1] (a2) {};
    \path[->] node[format, right=0cm and 3cm of a2] (s2) {Internet}
    (a2) edge node {$\lambda_{2}$} (s2);

    \path[->] node[below of=a2] (a3) {};
    \path[->] node[format, right=0cm and 3cm of a3] (s3) {FAX}
    (a3) edge node {$\lambda_{3}$} (s3);

    \path[->] node[format, right=0cm and 4cm of s1] (s4) {Facturas}
    (s1) edge node {1} (s4)
    (s2) edge node {2} (s4)
    (s3) edge node {3} (s4);

    \path[->] node[format, below of=s4] (s5) {Nuevos Clientes}
    (s1) edge node {1} (s5)
    (s2) edge node {2} (s5)
    (s3) edge node {3} (s5);

    \path[->] node[format, below of=s5] (s6) {Reclamaciones}
    (s1) edge node {1} (s6)
    (s2) edge node {2} (s6)
    (s3) edge node {3} (s6)
    (s6) edge node {4} (s5);

    \path[->] node[format, below of=s6] (s7) {Ser. Técnico}
    (s1) edge node {1} (s7)
    (s2) edge node {2} (s7)
    (s3) edge node {3} (s7)
    (s6) edge node {4} (s7);

    \path[->] node[right=0cm and 3cm of s4] (f1) {}
    (s4) edge node {} (f1);

    \path[->] node[right=0cm and 3cm of s5] (f2) {}
    (s5) edge node {} (f2);

    \path[->] node[right=0cm and 3cm of s6] (f3) {}
    (s6) edge node {} (f3);

    \path[->] node[right=0cm and 3cm of s7] (f4) {}
    (s7) edge node {} (f4);

  \end{tikzpicture}}
  \caption{Diagrama TQM del sistema}
\end{figure}
